% from https://texample.net/tikz/examples/free-body-diagrams/

\documentclass[crop,tikz]{standalone}

\usepackage{pst-eucl}


\usepackage{tikz}
\usetikzlibrary{scopes}
\usetikzlibrary{calc}
\usetikzlibrary{decorations.pathreplacing}



\begin{document}

\def\iangle{-20} % pedal angle
\def\lcangle{15} % angle

\def\llength{8} % angle

\def\down{-90}
\def\up{90}
\def\arcr{1cm} % Radius of the arc used to indicate angles
\def\R{0.33}
\def\a{2.5}

\tikzstyle{pivot}=[thin,top color=gray!50,bottom color=gray!90!black,shading angle=20]


\def\centerarc[#1](#2)(#3:#4:#5)% Syntax: [draw options] (center) (initial angle:final angle:radius)
    { \draw[#1] ($(#2)+({#5*cos(#3)},{#5*sin(#3)})$) arc (#3:#4:#5); }
    

\begin{tikzpicture}[
    force/.style={>=latex,draw=blue,fill=blue},
    axis/.style={densely dashed,gray,font=\small},
    M/.style={rectangle,draw,fill=lightgray,minimum size=0.5cm,thin},
    m/.style={rectangle,draw=black,fill=lightgray,minimum size=0.3cm,thin},
    plane/.style={draw=black,fill=blue!10},
    string/.style={draw=red, thick},
    pulley/.style={thick},
]

    %% Free body diagram of M
    \begin{scope}[rotate=\iangle]
        \node[M,transform shape] (M) {};
        % Draw axes and help lines

        % define pivots
        \coordinate (A) at (0,0);
        \coordinate (B) at (0,5);
        \coordinate (C) at ($(B)+(\lcangle:\llength)$);
        \coordinate (D) at (0,8);

        % x = cos(\lcangle) * \llength + b * cos(phi)
        % y = sin(\lcangle) * \llength
        % phi = 90+\iangle = 70
        \coordinate (E) at (7.7274, 2.07055);


        

        % draw connection lines between pivots
        \draw[line width=2pt] (A) -- (B);
        \draw[line width=2pt] (B) -- (D);
        \draw[line width=2pt] (B) -- (C);

        

            
        {[axis,->]
            

            % draw coordinate system @ pivot A
            \draw (A) -- ++ (0,2) node[right] {$+y$};
            \draw (A) -- ++(2,0) node[right] {$+x$};
            
            % draw pivots
            \draw[pivot] (A) circle (\R) node[scale=1.2] {$+$};
            \draw[pivot] (B) circle (\R) node[scale=1.2] {$+$};
            \draw[pivot] (C) circle (\R) node[scale=1.2] {$+$};
            \draw[pivot] (D) circle (\R) node[scale=1.2] {$+$};

            % draw force @D

            % draw vertical line @A
            

            
            % draw pedal angle @A
            %\draw[solid,shorten >=0.5pt] (\up-\iangle:\arcr)
            %    arc(\up-\iangle:\up:\arcr);
            %\node at (\up-0.5*\iangle:1.3*\arcr) {$\phi$};

            % draw angle gamma

            

            
        }

        % draw phi
        \centerarc[green,thick, ->](0,0)(-\iangle:89:2.5)
        \node[green] at ($(A)+(1.6,2.6)$) {$\phi$};
        
        % draw alpha
        \centerarc[red,thick, ->](0,0)(41:89:1)
        \node[red] at ($(A)+(0.3,0.6)$) {$\alpha$};

        % draw gamma
        \centerarc[red,thick, ->](B)(\down:\lcangle:1)
        \node[red] at ($(B)+(+0.5,-0.5)$) {$\gamma$};

        % draw beta
        \centerarc[red,thick, ->](C)(220:180+\lcangle:1)
        \node[red] at ($(C)+(-0.5,-0.3)$) {$\beta$};

        % draw alpha+
        \centerarc[green,thick, ->](0,0)(-\iangle:41:3.5)
        \node[green] at ($(A)+(3.5,1.9)$) {$\alpha^+$};

        % Forces
        {[force,->,line width=2pt]
            % Assuming that Mg = 1. The normal force will therefore be cos(alpha)
            %\draw (M.center) -- ++(0,{cos(\iangle)}) node[above right] {$N$};
            %\draw (M.west) -- ++(-1,0) node[left] {$f_R$};

            % draw pedal force
            \draw ($(D.east)+(-2,0)$) node[above] {$F_p$} -- ++(2,0) ;

            % draw pivot force @A
            \draw ($(A.east)+(-1,0)$) node[above] {$F_{A,x}$} -- ++(1,0) ;
            \draw ($(A.east)+(0,-1)$) node[left] {$F_{A,y}$} -- ++(0,1) ;

            % draw loadcell force
            \draw ($(B.east)+(0,0)$) -- ++(180+\lcangle:2) node[above] {$F_{l}$};
        }

    % draw helper line @B
    \draw[dashed] (B) -- ++ (4,0) node[right] {};

    \centerarc[green,thick, ->](B)(0:\lcangle:3)
    \node[green] at ($(B)+(+2.5,+0.3)$) {$\gamma^+$};
    
    
    \end{scope}
    
    % Draw gravity force. The code is put outside the rotated
    % scope for simplicity. No need to do any angle calculations. 
    %\draw[force,->] (M.center) -- ++(0,-1) node[below] {$Mg$};
    %%

    % draw helper line for angle
    %\draw[dashed] (A) -- ++ (0,4) node[right] {};
    %\draw[dashed] (A) -- ++ (9.7,0) node[right] {};

    %\draw[dashed] (C) -- ++ (0,-3.9) node[right] {};

    % https://stackoverflow.com/questions/1588568/how-to-get-one-component-of-a-tikz-pgf-coordinate
    \draw[dashed]
          let
            \p1=(A),
            \p2=(B),
            \p3=(C)
          in
            (\x1,\y1) -- (\x3, \y1) -- (\x3, \y3);

    % draw c_hor        
    \draw[thick,decorate,decoration={brace,amplitude=12pt}, gray] 
    let
            \p1=(A),
            \p2=(B),
            \p3=(C)
          in
    (\x3,\y1) -- (\x1, \y1) node[midway, below, xshift=0pt, yshift=-10pt,]{$c_\text{hor}$};


    % draw c_ver        
    \draw[thick,decorate,decoration={brace,amplitude=12pt}, gray] 
    let
            \p1=(A),
            \p2=(B),
            \p3=(C)
          in
    (\x3,\y3) -- (\x3, \y1) node[midway, below, xshift=30pt, yshift=10pt,]{$c_\text{vert}$};
    


    

    % draw horizontal and vertical line AC
    \draw[dashed] (A) -- (C);


    % draw length dimensions
    \draw[thick,decorate,decoration={brace,amplitude=12pt}, gray] (A.north west) -- (B.north east) node[midway, above, xshift=-15pt, yshift=0pt,]{$b$};

    \draw[thick,decorate,decoration={brace,amplitude=12pt}, gray] (B.north west) -- (D.north east) node[midway, above, xshift=-15pt, yshift=0pt,]{$d$};

    \draw[thick,decorate,decoration={brace,amplitude=12pt}, gray] (B.north west) -- (C.north east) node[midway, above, xshift=0pt, yshift=10pt,]{$a$};


    


\end{tikzpicture}

\end{document}